\documentclass[t,8pt]{beamer}
\input{config_styles/usepackages}
\input{config_styles/commands}

% Select theme
\usetheme{mystyle}

% Credentials
\title[Optimierung industrieller Prozesse]{Optimierung industrieller Prozesse}
\author{Gedeon, Fabian, Alex, Philip, Daniel}
\date{12.04.2017}
\institute{DHBW Mannheim}

% Number of levels in toc
\dwDeepToc{false}

% List of figures/tables
\dwLofLot{false}

\begin{document}

\dwPrintTitle{images/evolution}{images/logo_dhbw}
\dwPrintToc{1.2}

\dwSection{Betrachtete Algorithmen}

% ================================================================

\begin{dwHeaderFrame}{Genetischer Algorithmus}
	\begin{dwItemize}
		\item Funktionsweise ist der Natur entlehnt.
		\item Initialisierung einer Population.
		\item Bewertung der Individuen mithilfe einer Fitnessfunktion.
		\item \textbf{Operationen:}

		\begin{dwInnerItemize}
			\item Selektion (der Eltern)
			\item Crossover
			\item Mutation
			\item Ersetzung
		\end{dwInnerItemize}

		\item \textbf{Parameter:}

		\begin{dwInnerItemize}
			\item Popualtionsgröße \& Generationsanzahl
			\item Mutationsrate
			\item Crossoverrate
		\end{dwInnerItemize}

	\end{dwItemize}
\end{dwHeaderFrame}

% ================================================================

\begin{dwFrame}
	\dwHeader{Funktionsweise des Algorithmus}

	\dwFramedFigure{\includegraphics[scale=0.9]{images/genetic_algo}}{Funktionsweise des genetischen Algorithmus}{fig:genetci-algo}
\end{dwFrame}

% ================================================================

\begin{dwFrame}
	\dwHeader{Realisierung}

	\begin{dwItemize}
			\item Crossover-Strategien:

			\begin{dwInnerItemize}
				\item \textbf{Gewichteter Durchschnitt:} Berechnung des gewichteten Durchschnitts (basierend auf der Fitness 								der Eltern) für jedes Gen des Kindes.
				\item \textbf{Singlepoint Crossover:} Kind erhält zwei unterschiedliche Teile des Vektors der Eltern. Trennung 								an einem Punkt.
				\item \textbf{Multipoint Crossover:} Kind erhält beliebig viele unterschiedliche Teile des Vektors der Eltern. 														Trennung an beliebig vielen Punkten.
			\end{dwInnerItemize}

			\item Mutations-Strategien:

			\begin{dwInnerItemize}
				\item \textbf{Vertauschen zweier Parameter:} Mutationswahrscheinlichkeit für gesamtes Individuum. 										Vertauschen einer festen Anzahl von Genen.
				\item \textbf{Mutation mit fester Anzahl:} Mutationswahrscheinlichkeit für gesamtes Individuum.
							Zufällige Werte für eine feste Anzahl von Genen.
				\item \textbf{Mutation mit variabler Anzahl:} Mutationswahrscheinlichkeit für jedes Gene einzeln.
							Zufällige Werte für eine feste Anzahl von Genen.
				\item \textbf{Gauss Mutation:} Mutationswahrscheinlichkeit für gesamtes Individuum. Addieren eines Gauss-								verteilten Zufallswertes.
			\end{dwInnerItemize}
	\end{dwItemize}
\end{dwFrame}


% ================================================================

\begin{dwFrame}
	\dwHeader{Anpassungen}

	\begin{dwItemize}
		\item Anpassung der Mutationsrate über $e$-Funktion (Optimierung für 5000 Iterationen):\\
			\begin{equation*}
				e^{-x*0.0009-1}
			\end{equation*}
		\item Werte mit \glqq  not feasible\grqq{} werden mit einer Strafe von 1.000.000 für den Fitnesswert versehen.
		\item Versuch mit fixer Anzahl an Mutationen konsistente Änderungen und Data Space Exploration zu erreichen.
		\item Truncation Selection mit 25\% der Population
	\end{dwItemize}

	\vspace{2mm}

	\dwHeader{Parameter}

	\begin{dwItemize}
		\item Populationsgröße: $2^{13}$
		\item Generationsanzahl: 5000
		\item Mutationsrate (variabel): 0.08
		\item Mutationsrate (fix): 0.3 für 4 Individuen
	\end{dwItemize}

\end{dwFrame}

% ================================================================

\begin{dwHeaderFrame}{Particle Swarm Algorithmus}
	\begin{dwItemize}
		\item Vorbild dieses Algorithmus is das biologische Schwarmverhalten (z.B. Vögel oder Bienen)
		\item Die Population bewegt sich durch den Suchraum und orientiert sich am aktuell besten Individuum 
		\item Bewertung der Individuen mithilfe einer Fitnessfunktion.
		\item \textbf{Operationen:}

		\begin{dwInnerItemize}
			\item Initialisierung des Schwarms
			\item Fitnessfunktion
			\item Anpassen der Geschwindigkeit (Velocity)
			\item Schwarmoptimum bestimmen
		\end{dwInnerItemize}
		\item \textbf{Parameter:}
		\begin{dwInnerItemize}
			\item Populationsgröße
			\item Lokale Optimierung (C1)
			\item Globale Optimierung (C2)
		\end{dwInnerItemize}

	\end{dwItemize}

\end{dwHeaderFrame}

% ================================================================

\begin{dwFrame}
	\dwHeader{Funktionsweise des Algorithmus}

	\dwFramedFigure{\includegraphics[scale=0.35]{images/pso}}{Funktionsweise des Particle Swarm Algorithmus}{fig:pso}
\end{dwFrame}

% ================================================================

\begin{dwFrame}
	\dwHeader{Vorgehensweise}
	\dwFramedFigure{\includegraphics[scale=0.33]{images/pso_algorithm}}{Vorgehensweise des Particle Swarm Algorithmus in Pseudo Code}{fig:pso}

\end{dwFrame}


% ================================================================

\begin{dwFrame}
	\dwHeader{Parameter}

	\begin{dwItemize}
		\item Populationsgröße: $10^{4}$
		\item Iterationen: 1000 (Abbruchbedingung)
		\item C1: 1
		\item C2: 2
	\end{dwItemize}

\end{dwFrame}

% ================================================================


\dwSection{Erkenntnisse}
\begin{dwHeaderFrame}{Genetischer Algorithmus}
	\begin{dwItemize}
		\item Gute Näherung der Testfunktionen von Stiblinski-Tang und Rastrigin.
		\item Schlechte Näherung der Testfunktion von Rosenbrock.
		\item Gute Näherung der unbekannten Funktion
		\item Ergebnisse: \url{https://github.com/DaWe1992/OIP/blob/master/Results.md}
		\item Keine Eignung für alle Optimierungsprobleme
	\end{dwItemize}
\end{dwHeaderFrame}

% ================================================================

\begin{dwHeaderFrame}{Particle Swarm Algorithmus}
	\begin{dwItemize}
		\item Nachvollziehbares Vorgehen
		\item Gute Näherung der Testfunktionen von Rosenbrock
		\item Gute Näherung der unbekannten Funktion
		\item Ergebnisse: \url{https://github.com/DaWe1992/OIP/blob/master/Results.md}

		\item Im Gegensatz zum GA aktualisieren die Partikel ihre Kennzahlen selbstständig und verfügen über einen internen Speicher für das lokale Optimum
		\item Orientierung im Gegensatz zum GA nur an einem Indikator (Global Best) -> one way information sharing mechanism
	\end{dwItemize}
\end{dwHeaderFrame}

% ================================================================

\dwSection{Kommunikation}

\begin{dwHeaderFrame}{Anbindung an RabbitMQ}
	\begin{dwItemize}
		\item RabbitMQ Client kapselt Funktionalität von Sender und Receiver.
		\item RabbitMQ Client bietet \glqq Send and Wait\grqq{} $\Longrightarrow$ Blockierender Aufruf.
		\item Wiederverwendbarkeit für sämtlich Algorithmen.
	\end{dwItemize}

	\vspace{2mm}
	\dwHeader{Technische Limitation}

	\begin{dwItemize}
		\item Maximales Senden von 20k Nachrichten pro Sekunde.
	\end{dwItemize}

	\begin{center}
		\includegraphics[scale=0.4]{images/rabbitmq}
	\end{center}

\end{dwHeaderFrame}

% ================================================================

\end{document}
