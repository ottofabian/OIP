\documentclass[t,8pt]{beamer}
\input{config_styles/usepackages}
\input{config_styles/commands}

% Select theme
\usetheme{mystyle}

% Credentials
\title[Musterthema]{Optimierung industrieller Prozesse}
\author{Gedeon, Fabian, Alex, Philip, Daniel}
\date{12.04.2017}
\institute{DHBW Mannheim}

% Number of levels in toc
\dwDeepToc{false}

% List of figures/tables
\dwLofLot{false}

\begin{document}

\dwPrintTitle{images/evolution}{images/logo_dhbw}
\dwPrintToc{1.2}

\dwSection{Betrachtete Algorithmen}

% ================================================================

\begin{dwHeaderFrame}{Genetischer Algorithmus}
	\begin{dwItemize}
		\item Funktionsweise ist der Natur entlehnt.
		\item Initialisierung einer Population.
		\item Bewertung der Individuen mithilfe einer Fitnessfunktion.
		\item \textbf{Operationen:}

		\begin{dwInnerItemize}
			\item Selektion (der Eltern)
			\item Crossover
			\item Mutation
			\item Ersetzung
		\end{dwInnerItemize}

		\item \textbf{Parameter:}

		\begin{dwInnerItemize}
			\item Popualtionsgröße \& Generationsanzahl
			\item Mutationsrate
			\item Crossoverrate
		\end{dwInnerItemize}

	\end{dwItemize}
\end{dwHeaderFrame}

% ================================================================

\begin{dwFrame}
	\dwHeader{Funktionsweise des Algorithmus}

	\dwFramedFigure{\includegraphics[scale=0.9]{images/genetic_algo}}{Funktionsweise des genetischen Algorithmus}{fig:genetci-algo}
\end{dwFrame}

% ================================================================

\begin{dwFrame}
	\dwHeader{Realisierung}

	\begin{dwItemize}
			\item Crossover-Strategien:

			\begin{dwInnerItemize}
				\item \textbf{Gewichteter Durchschnitt:} Berechnung des gewichteten Durchschnitts (basierend auf der Fitness 								der Eltern) für jedes Gen des Kindes.
				\item \textbf{Singlepoint Crossover:} Kind erhält zwei unterschiedliche Teile des Vektors der Eltern. Trennung 								an einem Punkt.
				\item \textbf{Multipoint Crossover:} Kind erhält beliebig viele unterschiedliche Teile des Vektors der Eltern. 														Trennung an beliebig vielen Punkten.
			\end{dwInnerItemize}

			\item Mutations-Strategien:

			\begin{dwInnerItemize}
				\item \textbf{Vertauschen zweier Parameter:} Mutationswahrscheinlichkeit für gesamtes Individuum. 										Vertauschen einer festen Anzahl von Genen.
				\item \textbf{Mutation mit fester Anzahl:} Mutationswahrscheinlichkeit für gesamtes Individuum.
							Zufällige Werte für eine feste Anzahl von Genen.
				\item \textbf{Mutation mit variabler Anzahl:} Mutationswahrscheinlichkeit für jedes Gene einzeln.
							Zufällige Werte für eine feste Anzahl von Genen.
				\item \textbf{Gauss Mutation:} Mutationswahrscheinlichkeit für gesamtes Individuum. Addieren eines Gauss-								verteilten Zufallswertes.
			\end{dwInnerItemize}

	\end{dwItemize}
\end{dwFrame}


% ================================================================

\begin{dwFrame}
	\dwHeader{Anpassungen}

	\begin{dwItemize}
		\item Anpassung der Mutationsrate über $e$-Funktion (Optimierung für 5000 Iterationen):\\
			\begin{equation*}
				e^{-x*0.0009-1}
			\end{equation*}
		\item Werte mit \glqq  not feasible\grqq{} werden mit einer Strafe von 1.000.000 für den Fitnesswert versehen.
		\item Versuch mit fixer Anzahl an Mutationen konsistente Änderungen und Data Space Exploration zu erreichen.
		\item Truncation Selection mit 25\% der Population
	\end{dwItemize}

	\vspace{2mm}

	\dwHeader{Parameter}

	\begin{dwItemize}
		\item Populationsgröße: $2^{13}$
		\item Generationsanzahl: 5000
		\item Mutationsrate (variabel): 0.08
		\item Mutationsrate (fix): 0.3 für 4 Individuen
	\end{dwItemize}

\end{dwFrame}

% ================================================================

\begin{dwHeaderFrame}{Particle Swarm Algorithmus}
	\begin{dwItemize}
		\item Mehrere Partikel
	\end{dwItemize}
\end{dwHeaderFrame}

% ================================================================

\dwSection{Erkenntnisse}
\begin{dwHeaderFrame}{Genetischer Algorithmus}
	\begin{dwItemize}
		\item Gute Näherung der Testfunktionen von Stiblinski-Tang und Rastrigin.
		\item Schlechte Näherung der Testfunktion von Rosenbrock.
		\item Gute Näherung der unbekannten Funktion
		\item Ergebnisse: \url{https://github.com/DaWe1992/OIP/blob/master/Results.md}
		\item Keine Eignung für alle Optimierungsprobleme
	\end{dwItemize}
\end{dwHeaderFrame}

% ================================================================

\begin{dwHeaderFrame}{Particle Swarm Algorithmus}
	\begin{dwItemize}
		\item Gute Näherung der Testfunktionen von Rosenbrock
		\item Gute Näherung der unbekannten Funktion
	\end{dwItemize}
\end{dwHeaderFrame}

% ================================================================

\dwSection{Kommunikation}

\begin{dwHeaderFrame}{Anbindung an RabbitMQ}
	\begin{dwItemize}
		\item RabbitMQ Client kapselt Funktionalität von Sender und Receiver.
		\item RabbitMQ Client bietet \glqq Send and Wait\grqq{} $\Longrightarrow$ Blockierender Aufruf.
		\item Wiederverwendbarkeit für sämtlich Algorithmen.
	\end{dwItemize}

	\vspace{2mm}
	\dwHeader{Technische Limitation}

	\begin{dwItemize}
		\item Maximales Senden von 20k Nachrichten pro Sekunde.
	\end{dwItemize}

	\begin{center}
		\includegraphics[scale=0.4]{images/rabbitmq}
	\end{center}

\end{dwHeaderFrame}

% ================================================================

\end{document}
